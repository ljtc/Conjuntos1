\documentclass[12pt]{article}
\usepackage[spanish,mexico]{babel}
\usepackage[letterpaper, DIV=classic]{typearea}
\usepackage[T1]{fontenc}
\usepackage{gfsartemisia}
\let\iint\relax
\let\iiint\relax
\let\iiiint\relax
\let\idotsint\relax
\let\openbox\relax
\usepackage{mathtools}
\usepackage{eulervm}
\usepackage{amssymb}
\usepackage{amsthm}
\usepackage{mathrsfs}
\usepackage{xsim}
\usepackage{tasks}
\usepackage{tikz-cd}
\usepackage{titlesec} %Para cambiar estilo de secciones,subsecciones,etc.
\usepackage{titling} %Para cambiar estilo de título
\usepackage{dashrule} %Para hacer líneas punteadas

\usepackage{multicol}
\usepackage[shortlabels]{enumitem}
\settasks{label=\textsc{\roman*})}

\SetExerciseParameters{exercise}{
  exercise-name = EJ,
  exercise-heading = \paragraph
}

\DeclareExerciseTagging{teca}
\DeclareExerciseTagging{par}
\DeclareExerciseTagging{tezfc}
\DeclareExerciseTagging{parzfc}
\xsimsetup{collect}

\DeclareExerciseCollection[par=2]{ca}
\DeclareExerciseCollection[parzfc=2]{zfc}

% Para escribir el "tal que" de los conjuntos
\providecommand\st{\;|\;}

\newcommand\SetSymbol[1][]{%
    \nonscript\:#1\vert
    \allowbreak
    \nonscript\:
    \mathopen{}}
    \DeclarePairedDelimiterX\Set[1]\{\}{%
    \renewcommand\st{\SetSymbol[\delimsize]}
    #1
}
    \DeclarePairedDelimiterX\Class[1]\langle\rangle{%
    \renewcommand\st{\SetSymbol[\delimsize]}
    #1
}


%Comandos que utilizamos
\DeclarePairedDelimiter{\name}{\ulcorner}{\urcorner}
\newcommand{\ev}{\mathrm{ev}}
\newcommand{\topos}[1]{\mathscr{#1}}
\newcommand{\cat}[1]{\mathbf{#1}}
\newcommand{\op}{{}^{\mathrm{op}}}
\newcommand{\leqop}{\leq^{\mathrm{op}}}
\newcommand{\set}[1]{\{#1\}}
\newcommand{\ms}[1]{\mathscr{#1}}
\renewcommand{\emptyset}{\varnothing}
\newcommand{\pts}[1]{%
  \ifthenelse{\equal{#1}{1}}{\hfill \textbf{(#1 pt)}}{\hfill\textbf{(#1 pts)}}
}
\newcommand{\noloc}{%
  \nobreak
  \mspace{6mu plus 1mu}
  {:}
  \nonscript\mkern-\thinmuskip
  \mathpunct{}
  \mspace{2mu}
}


%Estilo de secciones
\titleformat*{\section}{\raggedleft\bfseries\Large}
\titleformat*{\subsection}{\raggedleft\bfseries\large}

%Estilo de título
\pretitle{\begin{center}\fontfamily{lmdh}\huge\bfseries}
\posttitle{\par\end{center}\vskip 0.5em}

\begin{document}

\section*{Conjuntos Abstractos}
Los ejercicios de esta sección se deben resolver en la categoría de
conjuntos abstractos, \(\topos{S}\), a menos que se indique lo contrario.

\collectexercises{ca}
\xsimsetup{par=2}
%Ejercicios para el primer parcial

\begin{exercise}[teca=1]
  Demuestra que \(f\colon A\to B\) es mono si y sólo si \(f\) es 
  inyectiva.
\end{exercise}

\begin{exercise}
  Una flecha \(f\colon A\to B\) es constante si se puede factorizar a
  través del terminal, es decir, existe \(b\colon 1\to B\) que hace
  conmutar al siguiente diagrama
  \begin{equation*}
      \begin{tikzcd}[ampersand replacement=\&] 
          A\ar{rr}{f}\ar{dr}[swap]{!_A} \&\& B\\
          \& 1\mathrlap{.}\ar{ur}[swap]{b}
      \end{tikzcd}
  \end{equation*}
  Muestra que \(f\colon A\to B\) es constante si y sólo si para
  cualesquiera \(a_1,a_2\colon 1\to A\) se satisface \(fa_1 = fa_2\).
\end{exercise}

\begin{exercise}
  Sean \(f\colon A\to B\) y \(g\colon B\to C\) flechas. Demuestra lo
  siguiente:
  \begin{tasks}(2)
      \task Si \(f\) y \(g\) son epi, entonces \(gf\) es epi.
      \task Si \(gf\) es epi, entonces \(g\) es epi.
  \end{tasks}
\end{exercise}

\begin{exercise}
  Sea \(\cat{P}\) la categoría generada por el orden parcial \((P,\leq)\).
  Muestra que toda flecha en \(\cat{P}\) es mono y epi. Con esto, da un
  ejemplo de una categoría en la que no se cumple que mono y epi implica iso.
\end{exercise}

\begin{exercise}
  Dadas una categoría \(\cat{A}\) y un objeto \(A\in\cat{A}\), se define
  la categoría rebanada  \(\cat{A}/A\) mediante lo siguiente: los objetos
  son flechas en \(\cat{A}\) de la forma \(f\colon X\to A\) y dados
  dos objetos \(f\colon X\to A\) y \(g\colon Y\to A\), una flecha de
  \(f\) a \(g\) es una flecha \(h\colon X\to Y\) en \(\cat{A}\) tal que
  \(g = hf\). Demuestra que \(\cat{A}/A\) es una categoría.
\end{exercise}

\begin{exercise}
  Muestra que para cualquier objeto \(A\), la rebanada \(\topos{S}/A\)
  tiene objetos terminal e inicial.
\end{exercise}

\begin{exercise}[teca=1]
  Sea \(m\colon S\rightarrowtail A\) un subobjeto y considera su flecha
  característica \(\chi_m\colon A\to \Omega\). Demuestra que para
  cualquier elemento generalizado \(x\colon X\to A\) se satisface: \( x\in_A m \iff \chi_m x = v_X \), donde \(v_X\) es la composición de \(!_X\colon X\to 1\) con \(v\colon 1\to \Omega\).
\end{exercise}

\begin{exercise}
  Sean \(f\colon A\to B\) y \(n\colon T\rightarrowtail B\). Usa el lema
  del producto fibrado para encontrar la característica de la imagen
  inversa \(f^{-1}n\).
\end{exercise}


%Ejercicios para el segundo parcial
\begin{exercise}[par=2]
  Sea \(\cat{A}\) una categoría con productos fibrados. Demuestra que un
  producto fibrado de \(f\colon A\to C\leftarrow B\noloc g\) es
  único salvo iso.
\end{exercise}

\begin{exercise}[par=2, teca=2]
  Muestra que el clasificador de subobjetos \(\Omega\) es coseparador, es
  decir, dadas \(f,g\colon A\to B\) si para cualquier 
  \(\varphi\colon B\to\Omega\) el diagrama
  \begin{equation}\label{eq:hip}
    \begin{tikzcd}[ampersand replacement=\&]
      A\ar[shift left]{r}{f}\ar[shift right]{r}[swap]{g}\& B\ar{r}{\varphi}\& \Omega      
    \end{tikzcd}
  \end{equation}
  conmuta, entonces \(f = g\).
\end{exercise}
\begin{solution}
  Primero veamos que el enunciado es cierto cuando \(A=1\). Esto es, si
  suponemos que \(b_1,b_2\colon 1\to B\) son tales que
  \begin{equation}\label{eq:puntos}
    \begin{tikzcd}[ampersand replacement=\&]
      1\ar[shift left]{r}{b_1}\ar[shift right]{r}[swap]{b_2}\& B\ar{r}{\varphi}\& \Omega     
    \end{tikzcd}
  \end{equation}
  conmuta, entonces veamos que \(b_1 = b_2\).

  Como toda flecha que sale del terminal es mono y \(\Omega\) es clasificador de
  subobjetos, entonces \(b_1\) tiene una característica, digamos 
  \(\varphi\colon B\to\Omega\). Por la hipótesis en~\eqref{eq:puntos} y la propiedad
  universal del producto fibrado tenemos que el siguiente diagrama conmuta:
  \begin{equation*}
    \begin{tikzcd}[ampersand replacement=\&]
      1\ar[bend left=20]{rrd}{\id}\ar[bend right=20]{ddr}[swap]{b_2}
        \ar[dashed]{rd}\\[-2ex]
      \&[-2ex] 1\ar{r}{\id}\ar{d}[swap]{b_1} \& 1\ar{d}{v}\\
      \& A\ar{r}[swap]{\varphi} \& \Omega\mathrlap{.}
    \end{tikzcd}
  \end{equation*}
  Como la única flecha del terminal a sí mismo es la identidad, entonces
  \(b_1 = b_2\).

  Ahora sea \(A\) arbitrario y supongamos que \(f,g\colon A\to B\) son tales
  que para cualquier \(\varphi\colon B\to\Omega\) el diagrama~\eqref{eq:hip}
  conmuta. Para ver que \(f = g\) usaremos que \(1\) es separador, es decir,
  veremos que para cualquier \(a\colon 1\to A\) se satisface \(fa=ga\).
  Por la hipótesis sobre \(f\) y \(g\) tenemos que el siguiente diagrama
  conmuta:
  \begin{equation*}
    \begin{tikzcd}[ampersand replacement=\&]
      1\ar[shift left]{r}{fa}\ar[shift right]{r}[swap]{ga}\& B\ar{r}{\varphi}\& \Omega.    
    \end{tikzcd}
  \end{equation*}
  Así, por lo que hicimos antes podemos concluir que \(fa = ga\).
\end{solution}

\begin{exercise}[par=2]
  Sea \(\cat{A}\) una categoría localmente pequeña con coproductos. Demuestra que
  \begin{equation*}
    \cat{A}(A,C)\times\cat{A}(B,C)\cong\cat{A}(A+B,C).
  \end{equation*}
\end{exercise}

\begin{exercise}[par=2,teca=2]
  Sean \(\ev\colon A\times\Omega^A\to\Omega\), \(x\colon X\to A\) y 
  \(m\colon S\rightarrowtail A\). Además, considera la característica de \(m\) y
  su nombre en la exponencial, \(\name{\chi_m}\colon 1\to\Omega^A\). Muestra que 
  \(x\in_A m\) si y sólo si \(\ev(x\times\name{\chi_m}) = v_{X\times 1}\).
\end{exercise}
\begin{solution}[print=true]
  Primero consideramos el siguiente diagrama
  \begin{equation}\label{eq:diagrama}
    \begin{tikzcd}[ampersand replacement=\&]
      A\times\Omega^A\ar{rr}{\ev} \&\& \Omega\\
      A\times 1\ar{u}{\id\times\name{\chi_m}}[name=a]{}\ar{r}{p_A}
      \& A\ar{ru}[name=b]{\chi_m} \\[-2ex]
      \&\& 1\ar{uu}[swap,name=y]{v} \\[-2ex]
      X\times 1\ar{uu}{x\times\id}[name=xid]{}\ar{r}{p_X}
      \& X\ar{uu}[name=x]{x}\ar{ur}[swap]{!_X}
      \ar[phantom, from=a, to=b, "1"]
      \ar[phantom, from=xid, to=x, "2"]
      \ar[phantom, from=x, to=y, "3"]
    \end{tikzcd}
  \end{equation}
  La parte 1 conmuta por la definición de \(\name{\chi_m}\) y la parte 2 por
  definición de la flecha \(x\times\id\). Si el diagrama 3 conmuta, entonces el
  diagrama exterior es conmutativo. Viceversa, si el diagrama exterior es
  conmutativo, entonces el diagrama 3 conmuta. En efecto, para ver que 3 conmuta
  es suficiente ver que conmuta desde \(X\times 1\), ya que \(p_X\) es iso. Si
  seguimos el diagrama podemos obtener la conmutatividad que queremos. Las
  ecuaciones que lo muestran son:
  \begin{align*}
    v\,!x\,p_X &= \ev\,(x\times\name{\chi_m})\,(x\times\id)
    && \text{diagrama exterior}\\
    & = \chi_m\,p_A\,(x\times\id) && \text{parte 1}\\
    &= \chi_m\,x\,p_X && \text{parte 2.}
  \end{align*} 
  Con esto hemos concluido que 3 conmuta si y sólo si el exterior conmuta.

  Ahora, si suponemos que \(x\in_A m\), entonces existe \(h\colon X\to S\) que
  hace conmutar al siguiente diagrama
  \begin{equation}\label{eq:defchi}
    \begin{tikzcd}[ampersand replacement=\&]
      X\ar[bend left=20]{rrd}{!_X}\ar[bend right=20]{ddr}[swap]{x}
        \ar{rd}{h}\\[-2ex]
      \&[-2ex] S\ar{r}{!_S}\ar[tail]{d}[swap]{m} \& 1\ar{d}{v}\\
      \& A\ar{r}[swap]{\chi_m} \& \Omega\mathrlap{.}
    \end{tikzcd}
  \end{equation}
  De esto tenemos que la parte 3 de diagrama~\eqref{eq:diagrama} conmuta. Así, el
  exterior conmuta. Por lo tanto \(\ev(x\times\name{\chi_m})=v_{X\times 1}\).
  
  Por el lado contrario, si \(\ev(x\times\name{\chi_m})=v_{X\times 1}\),
  entonces el exterior del diagrama~\eqref{eq:diagrama} conmuta. Así, la parte 3
  del mismo diagrama es conmutativa. Esta parte es el exterior del
  diagrama~\eqref{eq:defchi}. Por lo que podemos usar la propiedad universal del
  producto fibrado para obtener la existencia de \(h\colon X\to S\) que hace
  conmutar el diagrama~\eqref{eq:defchi}. Por lo tanto, \(x\in_A m\). 
\end{solution}

\begin{exercise}[par=2]
  Demuestra las siguientes leyes exponenciales:
  \begin{tasks}(2)
    \task \(A^0=1\)
    \task \(A^1=A\)
    \task \(A^{B+C}\cong A^B\times A^C\)
    \task \(A^{B\times C}\cong (A^B)^C\)
  \end{tasks}
  \textit{Sugerencia: usa, sin demostrar, que la biyección generada por la exponencial es natural en todas las entradas y el lema de Grothendieck-Yoneda.}
\end{exercise}

\begin{exercise}[par=2]
  Considera la categoría de espacios vectoriales sobre un campo \(k\),
  \(\cat{Vect}\). Da un ejemplo que muestre que el dual de un espacio no es
  natural, es decir, que el siguiente diagrama no conmuta:
  \begin{equation*}
    \begin{tikzcd}[ampersand replacement=\&]
      V\ar{r}{()^*}\ar{d}[swap]{T} \& V^*\\
      W\ar{r}[swap]{()^*} \& W^*\mathrlap{.}\ar{u}[swap]{T^*}     
    \end{tikzcd}
  \end{equation*}
  Además, muestra que doble dual sí es natural, es decir, que el
  siguiente diagrama conmuta:
  \begin{equation*}
    \begin{tikzcd}[ampersand replacement=\&]
      V\ar{r}{()^{**}}\ar{d}[swap]{T} \& V^{**}\ar{d}{T^{**}}\\
      W\ar{r}[swap]{()^{**}} \& W^{**}\mathrlap{.}
    \end{tikzcd}
  \end{equation*}
\end{exercise}
\collectexercisesstop{ca}
\printcollection{ca}

\section*{ZFC}
Resuelvan los ejercicios de esta sección utilizando únicamente los axiomas de ZFC vistos en clase (aún NO se puede usar el axioma del infinito)

\collectexercises{zfc}
\begin{exercise}[tezfc=1]
  Demuestre las siguientes equivalencias o implicaciones. En cada inciso indique claramente qué axiomas de ZFC se utilizan durante la prueba.
   \begin{enumerate}[i)]
       \item El axioma de extensionalidad implica el enunciado \(\forall x \forall y ( \forall w (x \in w \leftrightarrow y \in w) \rightarrow x=y ) \).
       %\item El enunciado \(\forall x \forall y \exists p \forall w ( (w=x \lor w=y) \to w \in p ) \) es equivalente al axioma del par.
       \item El enunciado \(\forall x \exists p \forall w ( \forall z ( z \in x \to z \in w) \rightarrow w \in p )\) es equivalente al axioma de potencia.
       \item El enunciado \( \forall x \forall y \exists p \forall w ( w \in p \leftrightarrow (p \in x \lor p=y ) ) \) implica el axioma del par.
   \end{enumerate}
\end{exercise}

\begin{exercise}
  Los siguientes enunciados son versiones ``débiles'' de los axiomas de par y potencia, respectivamente. Demuestra que éstos son equivalentes a sus contrapartes, los axiomas ``no débiles'' del par y potencia, respectivamente. En cada inciso indica claramente cuáles axiomas de ZFC se utilizan para probar la equivalencia.
  \begin{enumerate}[i)]
      \item \(\forall x \forall y \exists p \forall w ( (w=x \lor w=y) \to w \in p ) \) es al axioma débil del par.
      \item \(\forall x \exists p \forall w ( \forall z ( z \in x \to z \in w) \rightarrow w \in p )\) es el axioma débil del potencia.
  \end{enumerate}
\end{exercise}

\begin{exercise}
  Sea \(A\) un conjunto. Da condiciones necesarias y suficientes sobre cómo debe ser \(A\) para que la cualesquiera \(\Set{x \st \forall z \forall y((z \in A \land y \in z) \to x \in y)}\) sea conjunto.
\end{exercise}

\begin{exercise}
  Para cada inciso escribe una fórmula de primer orden en la teoría de conjuntos que describa el correspondiente concepto. En las fórmulas \textit{únicamente} se pueden utilizar símbolos lógicos, paréntesis, cuantificadores, variables y el símolo `\(\in\)'; sin abreviaturas de lenguaje como `\(\subseteq\)', `\(x=\emptyset\)', `\(x=\set{y}\)', etcétera. Se puede abreviar una fórmuila \textit{sólo si} ésta ya se escribió en un inciso anterior.
  \begin{multicols}{2}
      \begin{enumerate}[i)]
          \item \(x\) es el conjunto par de \(y\) y \(z\).
          \item \(x\) es el par ordenado de \(y\) y \(z\).
          \item \(x\) es par ordenado.
          \item \(x\) es la primera entrada del par ordenado \(y\).
          \item \(x\) es la segunda entrada de un par ordenado.
          \item \(x\) es una relación.
          \item \(x\) es el dominio de la relación \(y\).
          \item \(x\) es el campo de la relación \(y\).
          \item \(x=0\).
          \item \(x=1\).
          \item \(x=4\).
          \item \(x\) es la intersección de \(y\).
          \item \(x\) es elemento de la intersección de \(y\).
          \item \(x\) es la intersección de la intersección de \(y\).
      \end{enumerate}
  \end{multicols}
  Sólo hay que dar las fórmula, no es necesario ningún tipo de justificación.
\end{exercise}

\begin{exercise}
  Es un hecho que todas las colecciones de este ejercicio son conjuntos. Demuestra o refuta (con un contraejemplo) cuatro de los siguientes incisos, prueba todas tus afirmaciones.
  \begin{multicols}{2}
      \begin{enumerate}[i)]
          \item \( \bigcup \set{\set{x},\set{y}}=\set{x,y} \).
          \item \( \bigcup\bigcup\bigcup\set{\set{\set{x}}} = x \).
          \item \( \bigcup \set{x}=\emptyset \) y $x=\emptyset$ son equivalentes.
          \item Se da la igualdad \( (x,y)=(a,b) \) únicamente si \(x=a\) y \(y=b\).
          \item \( \set{x,y}=\set{a,b} \) si y sólo si $x=a$ y $y=b$.
          \item \( \ms{P}(\emptyset)=\set{\emptyset} \).
          \item \( \set{\emptyset,\set{\emptyset}} \notin \set{\emptyset,\set{\emptyset}} \).
          \item Se tiene \( \set{\set{x},\set{x,y},\set{x,y,z}} = \set{\set{a},\set{a,b},\set{a,b,c}} \) sólo cuando \(x=a\), \(y=b\) y \(z=c\).
          
      \end{enumerate}
  \end{multicols}
\end{exercise}

\begin{exercise}
  Determina cuales de las siguientes afirmaciones son verdaderas, justifica tu respuesta con una demostración o un contraejemplo. Demuestra todas tus afirmaciones.
  \begin{enumerate}[i)]
      \item Para todo conjunto \(x\) existe un conjunto \(y\) tal que \(x \not\subseteq y\)
      \item Para todo conjunto \(x\) existe un conjunto \(y\) tal que \(x \notin y\)
  \end{enumerate}
\end{exercise}

\begin{exercise}[tezfc=1]
  Todas las colecciones de este ejercicio son conjuntos. Prueba dos de los siguientes incisos:
  \begin{enumerate}[i)]
      \item \(x \subseteq \ms{P}(y)\) si y sólo si \(\bigcup x \subseteq y\).
      \item Si \(x \neq \emptyset\), entonces \(y \in \bigcap\Set{\ms{P}(a) \st a \in x}\) ocurre sólo si \(y \subseteq \bigcap x\).
      \item \( \bigcup \Set{ \ms{P}(a) \st a \in x } \subseteq \ms{P}(\bigcup x) \) pero no siempre \( \bigcup \Set{ \ms{P}(a) \st a \in x } \neq \ms{P}(\bigcup x) \).
      \item \( ( \bigcup x \big) \cap ( \bigcup y ) = \bigcup \Set{ a \cap b \st (a,b) \in x \times y } \).
  \end{enumerate}
\end{exercise}

\begin{exercise}
  Sean $X,Y, \ms{F}$ conjuntos tales que $\ms{F}\neq \emptyset$ y $f\colon X\to Y$ una función. Demuestra que las siguientes clases son conjuntos
  \begin{enumerate}[i)]
      \item $\Class{\bigcup \ms{G}\st \ms{G}\in \ms{F}}$
      \item $\Class{x \st \exists v \exists w \exists y \exists z (v\in \ms{F} \wedge w\in v \wedge y \in w \wedge z \in y \wedge x \in z)}$
      \item $\Class{x \st \forall \ms{G} \in \ms{F} \exists A \in \ms{G} (x\in A)}$
      \item $\Class{\ms{P}(A)\st A\in \ms{F}}$
      \item $\Class{A\times \ms{P}(A)\st A\in \ms{F}}$
      \item $\Class{B\setminus(f[A])\st A\subseteq X \wedge B\in \ms{F}}$
  \end{enumerate}
\end{exercise}

\begin{exercise}[tezfc=1]
  Sean $x$ un conjunto y $f$ una función con dominio $x$. Prueba lo siguiente:
  \begin{enumerate}[i)]
      \item Si $A \in \ms{P}(\ms{P}(X))$ es no vacío, entonces $f[\bigcap A] \subseteq \bigcap \Set{f[a] \st a \in A} $.
      \item $f$ es inyectiva si y sólo si para cada $A \in \ms{P}(\ms{P}(X))$ no vacío se tiene que $\bigcap \Set{f[a] \st a \in A} \subseteq f[\bigcap A]$.
  \end{enumerate}
\end{exercise}
\collectexercisesstop{zfc}
\printcollection{zfc}

\end{document}