\documentclass[12pt]{article}
\usepackage[spanish,mexico]{babel}
\usepackage[letterpaper, DIV=classic]{typearea}
\usepackage[T1]{fontenc}
\usepackage{gfsartemisia}
\let\iint\relax
\let\iiint\relax
\let\iiiint\relax
\let\idotsint\relax
\let\openbox\relax
\usepackage{mathtools}
\usepackage{eulervm}
\usepackage{amssymb}
\usepackage{amsthm}
\usepackage{mathrsfs}
\usepackage{xsim}
\usepackage{tasks}
\usepackage{tikz-cd}
\usepackage{titlesec} %Para cambiar estilo de secciones,subsecciones,etc.
\usepackage{titling} %Para cambiar estilo de título
\usepackage{dashrule} %Para hacer líneas punteadas

\usepackage{multicol}
\usepackage[shortlabels]{enumitem}
\settasks{label=\textsc{\roman*})}

\SetExerciseParameters{exercise}{
  exercise-name = Ejercicio,
  exercise-heading = \paragraph
}

\DeclareExerciseTagging{teca}
\DeclareExerciseTagging{par}
\DeclareExerciseTagging{tezfc}
\xsimsetup{collect}

\DeclareExerciseCollection[teca=2]{ca}
\DeclareExerciseCollection[tezfc=2]{zfc}

% Para escribir el "tal que" de los conjuntos
\providecommand\st{}

\newcommand\SetSymbol[1][]{%
    \nonscript\:#1\vert
    \allowbreak
    \nonscript\:
    \mathopen{}}
    \DeclarePairedDelimiterX\Set[1]\{\}{%
    \renewcommand\st{\SetSymbol[\delimsize]}
    #1
}
    \DeclarePairedDelimiterX\Class[1]\langle\rangle{%
    \renewcommand\st{\SetSymbol[\delimsize]}
    #1
}


%Comandos que utilizamos
\newcommand{\topos}[1]{\mathscr{#1}}
\newcommand{\cat}[1]{\mathbf{#1}}
\newcommand{\op}{{}^{\mathrm{op}}}
\newcommand{\leqop}{\leq^{\mathrm{op}}}
\newcommand{\set}[1]{\{#1\}}
\newcommand{\ms}[1]{\mathscr{#1}}
\renewcommand{\emptyset}{\varnothing}
\newcommand{\pts}[1]{%
  \ifthenelse{\equal{#1}{1}}{\hfill \textbf{(#1 pt)}}{\hfill\textbf{(#1 pts)}}
}
\newcommand{\noloc}{%
  \nobreak
  \mspace{6mu plus 1mu}
  {:}
  \nonscript\mkern-\thinmuskip
  \mathpunct{}
  \mspace{2mu}
}


%Estilo de secciones
\titleformat*{\section}{\raggedleft\bfseries\Large}
\titleformat*{\subsection}{\raggedleft\bfseries\large}

%Estilo de título
\pretitle{\begin{center}\fontfamily{lmdh}\huge\bfseries}
\posttitle{\par\end{center}\vskip 0.5em}

\begin{document}

\section*{Conjuntos Abstractos}
Los ejercicios de esta sección se deben resolver en la categoría de
conjuntos abstractos, \(\topos{S}\), a menos que se indique lo contrario.

\collectexercises{ca}
\xsimsetup{teca=2}
\begin{exercise}[teca=1]
  Demuestra que \(f\colon A\to B\) es mono si y sólo si \(f\) es 
  inyectiva.
\end{exercise}

\begin{exercise}
  Una flecha \(f\colon A\to B\) es constante si se puede factorizar a
  través del terminal, es decir, existe \(b\colon 1\to B\) que hace
  conmutar al siguiente diagrama
  \begin{equation*}
      \begin{tikzcd}[ampersand replacement=\&] 
          A\ar{rr}{f}\ar{dr}[swap]{!_A} \&\& B\\
          \& 1\mathrlap{.}\ar{ur}[swap]{b}
      \end{tikzcd}
  \end{equation*}
  Muestra que \(f\colon A\to B\) es constante si y sólo si para
  cualesquiera \(a_1,a_2\colon 1\to A\) se satisface \(fa_1 = fa_2\).
\end{exercise}

\begin{exercise}
  Sean \(f\colon A\to B\) y \(g\colon B\to C\) flechas. Demuestra lo
  siguiente:
  \begin{tasks}(2)
      \task Si \(f\) y \(g\) son epi, entonces \(gf\) es epi.
      \task Si \(gf\) es epi, entonces \(g\) es epi.
  \end{tasks}
\end{exercise}

\begin{exercise}
  Sea \(\cat{P}\) la categoría generada por el orden parcial \((P,\leq)\).
  Muestra que toda flecha en \(\cat{P}\) es mono y epi. Con esto, da un
  ejemplo de una categoría en la que no se cumple que mono y epi implica iso.
\end{exercise}

\begin{exercise}
  Dadas una categoría \(\cat{A}\) y un objeto \(A\in\cat{A}\), se define
  la categoría rebanada  \(\cat{A}/A\) mediante lo siguiente: los objetos
  son flechas en \(\cat{A}\) de la forma \(f\colon X\to A\) y dados
  dos objetos \(f\colon X\to A\) y \(g\colon Y\to A\), una flecha de
  \(f\) a \(g\) es una flecha \(h\colon X\to Y\) en \(\cat{A}\) tal que
  \(g = hf\). Demuestra que \(\cat{A}/A\) es una categoría.
\end{exercise}

\begin{exercise}
  Muestra que para cualquier objeto \(A\), la rebanada \(\topos{S}/A\)
  tiene objetos terminal e inicial.
\end{exercise}

\begin{exercise}[teca=1]
  Sea \(m\colon S\rightarrowtail A\) un subobjeto y considera su flecha
  característica \(\chi_m\colon A\to \Omega\). Demuestra que para
  cualquier elemento generalizado \(x\colon X\to A\) se satisface: \( x\in_A m \iff \chi_m x = v_X \), donde \(v_X\) es la composición de \(!_X\colon X\to 1\) con \(v\colon 1\to \Omega\).
\end{exercise}

\begin{exercise}
  Sean \(f\colon A\to B\) y \(n\colon T\rightarrowtail B\). Usa el lema
  del producto fibrado para encontrar la característica de la imagen
  inversa \(f^{-1}n\).
\end{exercise}
\collectexercisesstop{ca}
\printcollection{ca}

\end{document}