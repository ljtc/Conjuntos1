\begin{exercise}[teca=1]
  Demuestra que \(f\colon A\to B\) es mono si y sólo si \(f\) es 
  inyectiva.
\end{exercise}
\begin{solution}
  Supongamos que \(f\colon A\to B\) es mono y consideremos dos elementos
  globales \(a_1,a_2\colon 1\to A\) tales que el siguiente diagrama conmuta
  \begin{equation*}
    \begin{tikzcd}[ampersand replacement=\&]
      1\ar[shift left]{r}{a_1}\ar[shift right]{r}[swap]{a_2}\& A\ar{r}{f}\& B.
    \end{tikzcd}
  \end{equation*}
  Como \(f\) es mono se sigue que \(a_1=a_2\). Por lo tanto, \(f\) es inyectiva.

  Supongamos ahora que \(f\colon A\to B\) es inyectiva y supongamos que el
  siguiente diagrama conmuta
  \begin{equation}\label{eq:mono}
    \begin{tikzcd}[ampersand replacement=\&]
      T\ar[shift left]{r}{x}\ar[shift right]{r}[swap]{y}\& A\ar{r}{f}\& B.
    \end{tikzcd}
  \end{equation}
  Para mostrar que \(x=y\) usamos que \(1\) es separador. Así, tomemos un
  elemento global \(t\colon 1\to T\) y veamos que \(xt=yt\). Como el diagrama
  en~\eqref{eq:mono} conmuta, se sigue que
  \begin{equation*}
    \begin{tikzcd}[ampersand replacement=\&]
      1\ar{r}{t} \& T\ar[shift left]{r}{x}\ar[shift right]{r}[swap]{y}\& A\ar{r}{f}\& B
    \end{tikzcd}
    =
    \begin{tikzcd}[ampersand replacement=\&]
      1\ar[shift left]{r}{xt}\ar[shift right]{r}[swap]{yt}\& A\ar{r}{f}\& B.
    \end{tikzcd}
  \end{equation*}
  conmuta. Como \(f\) es inyectiva, se sigue que \(xt=yt\). Por lo tanto,
  \(x=y\) y así \(f\) es mono.
\end{solution}

\begin{exercise}
  Una flecha \(f\colon A\to B\) es constante si se puede factorizar a
  través del terminal, es decir, existe \(b\colon 1\to B\) que hace
  conmutar al siguiente diagrama
  \begin{equation*}
      \begin{tikzcd}[ampersand replacement=\&] 
          A\ar{rr}{f}\ar{dr}[swap]{!_A} \&\& B\\
          \& 1\mathrlap{.}\ar{ur}[swap]{b}
      \end{tikzcd}
  \end{equation*}
  Muestra que \(f\colon A\to B\) es constante si y sólo si para
  cualesquiera \(a_1,a_2\colon 1\to A\) se satisface \(fa_1 = fa_2\).
\end{exercise}

\begin{exercise}
  Sean \(f\colon A\to B\) y \(g\colon B\to C\) flechas. Demuestra lo
  siguiente:
  \begin{tasks}(2)
      \task Si \(f\) y \(g\) son epi, entonces \(gf\) es epi.
      \task Si \(gf\) es epi, entonces \(g\) es epi.
  \end{tasks}
\end{exercise}

\begin{exercise}
  Sea \(\cat{P}\) la categoría generada por el orden parcial \((P,\leq)\).
  Muestra que toda flecha en \(\cat{P}\) es mono y epi. Con esto, da un
  ejemplo de una categoría en la que no se cumple que mono y epi implica iso.
\end{exercise}

\begin{exercise}
  Dadas una categoría \(\cat{A}\) y un objeto \(A\in\cat{A}\), se define
  la categoría rebanada  \(\cat{A}/A\) mediante lo siguiente: los objetos
  son flechas en \(\cat{A}\) de la forma \(f\colon X\to A\) y dados
  dos objetos \(f\colon X\to A\) y \(g\colon Y\to A\), una flecha de
  \(f\) a \(g\) es una flecha \(h\colon X\to Y\) en \(\cat{A}\) tal que
  \(g = hf\). Demuestra que \(\cat{A}/A\) es una categoría.
\end{exercise}

\begin{exercise}
  Muestra que para cualquier objeto \(A\), la rebanada \(\topos{S}/A\)
  tiene objetos terminal e inicial.
\end{exercise}

\begin{exercise}[teca=1]
  Sea \(m\colon S\rightarrowtail A\) un subobjeto y considera su flecha
  característica \(\chi_m\colon A\to \Omega\). Demuestra que para
  cualquier elemento generalizado \(x\colon X\to A\) se satisface: \( x\in_A m \iff \chi_m x = v_X \), donde \(v_X\) es la composición de \(!_X\colon X\to 1\) con \(v\colon 1\to \Omega\).
\end{exercise}
\begin{solution}
  Supongamos que \(x\in_A m\), es decir, existe \(h\colon X\to S\) tal que
  el siguiente diagrama conmuta
  \begin{equation*}
    \begin{tikzcd}[ampersand replacement=\&]
      X\ar{rr}{h}\ar{dr}[swap]{x} \&\& S\ar[tail]{dl}{m}\\
      \& A\mathrlap{.}    
    \end{tikzcd}
  \end{equation*}
  Con esto la igualdad que queremos se sigue de la conmutatividad del siguiente
  diagrama
  \begin{equation*}
    \begin{tikzcd}[ampersand replacement=\&]
      X\ar[bend left=30]{rrd}{!_X}\ar{rd}{h}\ar[bend right=30]{ddr}[swap]{x}\\[-2ex]
      \&[-2ex] S\ar{r}{!_S}\ar[tail]{d}[swap]{m} \& 1\ar{d}{v}\\
      \& A\ar{r}[swap]{\chi_m} \& \Omega\mathrlap{.}
    \end{tikzcd}
  \end{equation*}

  Ahora supongamos que \(\chi_m x = v_X\). Esto significa que el cuadrado
  exterior del siguiente diagrama conmuta y por la propiedad universal del
  producto fibrado existe \(h\colon X\to S\) que hace conmutar al triangulo de
  la derecha del siguiente diagrama 
  \begin{equation*}
    \begin{tikzcd}[ampersand replacement=\&]
      X\ar[bend left=30]{rrd}{!_X}\ar[dashed]{rd}{h}\ar[bend right=30]{ddr}[swap]{x}\\[-2ex]
      \&[-2ex] S\ar{r}{!_S}\ar[tail]{d}[swap]{m} \& 1\ar{d}{v}\\
      \& A\ar{r}[swap]{\chi_m} \& \Omega\mathrlap{.}
    \end{tikzcd}
  \end{equation*}
  Por lo tanto, \(x\in_A m\).
\end{solution}

\begin{exercise}
  Sean \(f\colon A\to B\) y \(n\colon T\rightarrowtail B\). Usa el lema
  del producto fibrado para encontrar la característica de la imagen
  inversa \(f^{-1}n\).
\end{exercise}