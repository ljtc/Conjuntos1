\begin{exercise}[teca=1]
  Demuestra que \(f\colon A\to B\) es mono si y sólo si \(f\) es 
  inyectiva.
\end{exercise}

\begin{exercise}
  Una flecha \(f\colon A\to B\) es constante si se puede factorizar a
  través del terminal, es decir, existe \(b\colon 1\to B\) que hace
  conmutar al siguiente diagrama
  \begin{equation*}
      \begin{tikzcd}[ampersand replacement=\&] 
          A\ar{rr}{f}\ar{dr}[swap]{!_A} \&\& B\\
          \& 1\mathrlap{.}\ar{ur}[swap]{b}
      \end{tikzcd}
  \end{equation*}
  Muestra que \(f\colon A\to B\) es constante si y sólo si para
  cualesquiera \(a_1,a_2\colon 1\to A\) se satisface \(fa_1 = fa_2\).
\end{exercise}

\begin{exercise}
  Sean \(f\colon A\to B\) y \(g\colon B\to C\) flechas. Demuestra lo
  siguiente:
  \begin{tasks}(2)
      \task Si \(f\) y \(g\) son epi, entonces \(gf\) es epi.
      \task Si \(gf\) es epi, entonces \(g\) es epi.
  \end{tasks}
\end{exercise}

\begin{exercise}
  Sea \(\cat{P}\) la categoría generada por el orden parcial \((P,\leq)\).
  Muestra que toda flecha en \(\cat{P}\) es mono y epi. Con esto, da un
  ejemplo de una categoría en la que no se cumple que mono y epi implica iso.
\end{exercise}

\begin{exercise}
  Dadas una categoría \(\cat{A}\) y un objeto \(A\in\cat{A}\), se define
  la categoría rebanada  \(\cat{A}/A\) mediante lo siguiente: los objetos
  son flechas en \(\cat{A}\) de la forma \(f\colon X\to A\) y dados
  dos objetos \(f\colon X\to A\) y \(g\colon Y\to A\), una flecha de
  \(f\) a \(g\) es una flecha \(h\colon X\to Y\) en \(\cat{A}\) tal que
  \(g = hf\). Demuestra que \(\cat{A}/A\) es una categoría.
\end{exercise}

\begin{exercise}
  Muestra que para cualquier objeto \(A\), la rebanada \(\topos{S}/A\)
  tiene objetos terminal e inicial.
\end{exercise}

\begin{exercise}[teca=1]
  Sea \(m\colon S\rightarrowtail A\) un subobjeto y considera su flecha
  característica \(\chi_m\colon A\to \Omega\). Demuestra que para
  cualquier elemento generalizado \(x\colon X\to A\) se satisface: \( x\in_A m \iff \chi_m x = v_X \), donde \(v_X\) es la composición de \(!_X\colon X\to 1\) con \(v\colon 1\to \Omega\).
\end{exercise}

\begin{exercise}
  Sean \(f\colon A\to B\) y \(n\colon T\rightarrowtail B\). Usa el lema
  del producto fibrado para encontrar la característica de la imagen
  inversa \(f^{-1}n\).
\end{exercise}