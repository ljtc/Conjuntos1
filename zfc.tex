\begin{exercise}[tezfc=1]
  Demuestre las siguientes equivalencias o implicaciones. En cada inciso indique claramente qué axiomas de ZFC se utilizan durante la prueba.
   \begin{enumerate}[i)]
       \item El axioma de extensionalidad implica el enunciado \(\forall x \forall y ( \forall w (x \in w \leftrightarrow y \in w) \rightarrow x=y ) \).
       %\item El enunciado \(\forall x \forall y \exists p \forall w ( (w=x \lor w=y) \to w \in p ) \) es equivalente al axioma del par.
       \item El enunciado \(\forall x \exists p \forall w ( \forall z ( z \in x \to z \in w) \rightarrow w \in p )\) es equivalente al axioma de potencia.
       \item El enunciado \( \forall x \forall y \exists p \forall w ( w \in p \leftrightarrow (p \in x \lor p=y ) ) \) implica el axioma del par.
   \end{enumerate}
\end{exercise}

\begin{exercise}
  Los siguientes enunciados son versiones ``débiles'' de los axiomas de par y potencia, respectivamente. Demuestra que éstos son equivalentes a sus contrapartes, los axiomas ``no débiles'' del par y potencia, respectivamente. En cada inciso indica claramente cuáles axiomas de ZFC se utilizan para probar la equivalencia.
  \begin{enumerate}[i)]
      \item \(\forall x \forall y \exists p \forall w ( (w=x \lor w=y) \to w \in p ) \) es al axioma débil del par.
      \item \(\forall x \exists p \forall w ( \forall z ( z \in x \to z \in w) \rightarrow w \in p )\) es el axioma débil del potencia.
  \end{enumerate}
\end{exercise}

\begin{exercise}
  Sea \(A\) un conjunto. Da condiciones necesarias y suficientes sobre cómo debe ser \(A\) para que la cualesquiera \(\Set{x \st \forall z \forall y((z \in A \land y \in z) \to x \in y)}\) sea conjunto.
\end{exercise}

\begin{exercise}
  Para cada inciso escribe una fórmula de primer orden en la teoría de conjuntos que describa el correspondiente concepto. En las fórmulas \textit{únicamente} se pueden utilizar símbolos lógicos, paréntesis, cuantificadores, variables y el símolo `\(\in\)'; sin abreviaturas de lenguaje como `\(\subseteq\)', `\(x=\emptyset\)', `\(x=\set{y}\)', etcétera. Se puede abreviar una fórmuila \textit{sólo si} ésta ya se escribió en un inciso anterior.
  \begin{multicols}{2}
      \begin{enumerate}[i)]
          \item \(x\) es el conjunto par de \(y\) y \(z\).
          \item \(x\) es el par ordenado de \(y\) y \(z\).
          \item \(x\) es par ordenado.
          \item \(x\) es la primera entrada del par ordenado \(y\).
          \item \(x\) es la segunda entrada de un par ordenado.
          \item \(x\) es una relación.
          \item \(x\) es el dominio de la relación \(y\).
          \item \(x\) es el campo de la relación \(y\).
          \item \(x=0\).
          \item \(x=1\).
          \item \(x=4\).
          \item \(x\) es la intersección de \(y\).
          \item \(x\) es elemento de la intersección de \(y\).
          \item \(x\) es la intersección de la intersección de \(y\).
      \end{enumerate}
  \end{multicols}
  Sólo hay que dar las fórmula, no es necesario ningún tipo de justificación.
\end{exercise}

\begin{exercise}
  Es un hecho que todas las colecciones de este ejercicio son conjuntos. Demuestra o refuta (con un contraejemplo) cuatro de los siguientes incisos, prueba todas tus afirmaciones.
  \begin{multicols}{2}
      \begin{enumerate}[i)]
          \item \( \bigcup \set{\set{x},\set{y}}=\set{x,y} \).
          \item \( \bigcup\bigcup\bigcup\set{\set{\set{x}}} = x \).
          \item \( \bigcup \set{x}=\emptyset \) y $x=\emptyset$ son equivalentes.
          \item Se da la igualdad \( (x,y)=(a,b) \) únicamente si \(x=a\) y \(y=b\).
          \item \( \set{x,y}=\set{a,b} \) si y sólo si $x=a$ y $y=b$.
          \item \( \ms{P}(\emptyset)=\set{\emptyset} \).
          \item \( \set{\emptyset,\set{\emptyset}} \notin \set{\emptyset,\set{\emptyset}} \).
          \item Se tiene \( \set{\set{x},\set{x,y},\set{x,y,z}} = \set{\set{a},\set{a,b},\set{a,b,c}} \) sólo cuando \(x=a\), \(y=b\) y \(z=c\).
          
      \end{enumerate}
  \end{multicols}
\end{exercise}

\begin{exercise}
  Determina cuales de las siguientes afirmaciones son verdaderas, justifica tu respuesta con una demostración o un contraejemplo. Demuestra todas tus afirmaciones.
  \begin{enumerate}[i)]
      \item Para todo conjunto \(x\) existe un conjunto \(y\) tal que \(x \not\subseteq y\)
      \item Para todo conjunto \(x\) existe un conjunto \(y\) tal que \(x \notin y\)
  \end{enumerate}
\end{exercise}

\begin{exercise}[tezfc=1]
  Todas las colecciones de este ejercicio son conjuntos. Prueba dos de los siguientes incisos:
  \begin{enumerate}[i)]
      \item \(x \subseteq \ms{P}(y)\) si y sólo si \(\bigcup x \subseteq y\).
      \item Si \(x \neq \emptyset\), entonces \(y \in \bigcap\Set{\ms{P}(a) \st a \in x}\) ocurre sólo si \(y \subseteq \bigcap x\).
      \item \( \bigcup \Set{ \ms{P}(a) \st a \in x } \subseteq \ms{P}(\bigcup x) \) pero no siempre \( \bigcup \Set{ \ms{P}(a) \st a \in x } \neq \ms{P}(\bigcup x) \).
      \item \( ( \bigcup x \big) \cap ( \bigcup y ) = \bigcup \Set{ a \cap b \st (a,b) \in x \times y } \).
  \end{enumerate}
\end{exercise}

\begin{exercise}
  Sean $X,Y, \ms{F}$ conjuntos tales que $\ms{F}\neq \emptyset$ y $f\colon X\to Y$ una función. Demuestra que las siguientes clases son conjuntos
  \begin{enumerate}[i)]
      \item $\Class{\bigcup \ms{G}\st \ms{G}\in \ms{F}}$
      \item $\Class{x \st \exists v \exists w \exists y \exists z (v\in \ms{F} \wedge w\in v \wedge y \in w \wedge z \in y \wedge x \in z)}$
      \item $\Class{x \st \forall \ms{G} \in \ms{F} \exists A \in \ms{G} (x\in A)}$
      \item $\Class{\ms{P}(A)\st A\in \ms{F}}$
      \item $\Class{A\times \ms{P}(A)\st A\in \ms{F}}$
      \item $\Class{B\setminus(f[A])\st A\subseteq X \wedge B\in \ms{F}}$
  \end{enumerate}
\end{exercise}

\begin{exercise}[tezfc=1]
  Sean $x$ un conjunto y $f$ una función con dominio $x$. Prueba lo siguiente:
  \begin{enumerate}[i)]
      \item Si $A \in \ms{P}(\ms{P}(X))$ es no vacío, entonces $f[\bigcap A] \subseteq \bigcap \Set{f[a] \st a \in A} $.
      \item $f$ es inyectiva si y sólo si para cada $A \in \ms{P}(\ms{P}(X))$ no vacío se tiene que $\bigcap \Set{f[a] \st a \in A} \subseteq f[\bigcap A]$.
  \end{enumerate}
\end{exercise}