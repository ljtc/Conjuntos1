%Ejercicios para el primer parcial

\begin{exercise}[tezfc=1]
  Demuestre las siguientes equivalencias o implicaciones. En cada inciso indique claramente qué axiomas de ZFC se utilizan durante la prueba.
   \begin{enumerate}[i)]
       \item El axioma de extensionalidad implica el enunciado \(\forall x \forall y ( \forall w (x \in w \leftrightarrow y \in w) \rightarrow x=y ) \).
       %\item El enunciado \(\forall x \forall y \exists p \forall w ( (w=x \lor w=y) \to w \in p ) \) es equivalente al axioma del par.
       \item El enunciado \(\forall x \exists p \forall w ( \forall z ( z \in x \to z \in w) \rightarrow w \in p )\) es equivalente al axioma de potencia.
       \item El enunciado \( \forall x \forall y \exists p \forall w ( w \in p \leftrightarrow (p \in x \lor p=y ) ) \) implica el axioma del par.
   \end{enumerate}
\end{exercise}

\begin{exercise}
  Los siguientes enunciados son versiones ``débiles'' de los axiomas de par y potencia, respectivamente. Demuestra que éstos son equivalentes a sus contrapartes, los axiomas ``no débiles'' del par y potencia, respectivamente. En cada inciso indica claramente cuáles axiomas de ZFC se utilizan para probar la equivalencia.
  \begin{enumerate}[i)]
      \item \(\forall x \forall y \exists p \forall w ( (w=x \lor w=y) \to w \in p ) \) es al axioma débil del par.
      \item \(\forall x \exists p \forall w ( \forall z ( z \in x \to z \in w) \rightarrow w \in p )\) es el axioma débil del potencia.
  \end{enumerate}
\end{exercise}

\begin{exercise}
  Sea \(A\) un conjunto. Da condiciones necesarias y suficientes sobre cómo debe ser \(A\) para que la cualesquiera \(\Set{x \st \forall z \forall y((z \in A \land y \in z) \to x \in y)}\) sea conjunto.
\end{exercise}

\begin{exercise}
  Para cada inciso escribe una fórmula de primer orden en la teoría de conjuntos que describa el correspondiente concepto. En las fórmulas \textit{únicamente} se pueden utilizar símbolos lógicos, paréntesis, cuantificadores, variables y el símolo `\(\in\)'; sin abreviaturas de lenguaje como `\(\subseteq\)', `\(x=\emptyset\)', `\(x=\set{y}\)', etcétera. Se puede abreviar una fórmuila \textit{sólo si} ésta ya se escribió en un inciso anterior.
  \begin{multicols}{2}
      \begin{enumerate}[i)]
          \item \(x\) es el conjunto par de \(y\) y \(z\).
          \item \(x\) es el par ordenado de \(y\) y \(z\).
          \item \(x\) es par ordenado.
          \item \(x\) es la primera entrada del par ordenado \(y\).
          \item \(x\) es la segunda entrada de un par ordenado.
          \item \(x\) es una relación.
          \item \(x\) es el dominio de la relación \(y\).
          \item \(x\) es el campo de la relación \(y\).
          \item \(x=0\).
          \item \(x=1\).
          \item \(x=4\).
          \item \(x\) es la intersección de \(y\).
          \item \(x\) es elemento de la intersección de \(y\).
          \item \(x\) es la intersección de la intersección de \(y\).
      \end{enumerate}
  \end{multicols}
  Sólo hay que dar las fórmula, no es necesario ningún tipo de justificación.
\end{exercise}

\begin{exercise}
  Es un hecho que todas las colecciones de este ejercicio son conjuntos. Demuestra o refuta (con un contraejemplo) cuatro de los siguientes incisos, prueba todas tus afirmaciones.
  \begin{multicols}{2}
      \begin{enumerate}[i)]
          \item \( \bigcup \set{\set{x},\set{y}}=\set{x,y} \).
          \item \( \bigcup\bigcup\bigcup\set{\set{\set{x}}} = x \).
          \item \( \bigcup \set{x}=\emptyset \) y $x=\emptyset$ son equivalentes.
          \item Se da la igualdad \( (x,y)=(a,b) \) únicamente si \(x=a\) y \(y=b\).
          \item \( \set{x,y}=\set{a,b} \) si y sólo si $x=a$ y $y=b$.
          \item \( \ms{P}(\emptyset)=\set{\emptyset} \).
          \item \( \set{\emptyset,\set{\emptyset}} \notin \set{\emptyset,\set{\emptyset}} \).
          \item Se tiene \( \set{\set{x},\set{x,y},\set{x,y,z}} = \set{\set{a},\set{a,b},\set{a,b,c}} \) sólo cuando \(x=a\), \(y=b\) y \(z=c\).
          
      \end{enumerate}
  \end{multicols}
\end{exercise}

\begin{exercise}
  Determina cuales de las siguientes afirmaciones son verdaderas, justifica tu respuesta con una demostración o un contraejemplo. Demuestra todas tus afirmaciones.
  \begin{enumerate}[i)]
      \item Para todo conjunto \(x\) existe un conjunto \(y\) tal que \(x \not\subseteq y\)
      \item Para todo conjunto \(x\) existe un conjunto \(y\) tal que \(x \notin y\)
  \end{enumerate}
\end{exercise}

\begin{exercise}[tezfc=1]
  Todas las colecciones de este ejercicio son conjuntos. Prueba dos de los siguientes incisos:
  \begin{enumerate}[i)]
      \item \(x \subseteq \ms{P}(y)\) si y sólo si \(\bigcup x \subseteq y\).
      \item Si \(x \neq \emptyset\), entonces \(y \in \bigcap\Set{\ms{P}(a) \st a \in x}\) ocurre sólo si \(y \subseteq \bigcap x\).
      \item \( \bigcup \Set{ \ms{P}(a) \st a \in x } \subseteq \ms{P}(\bigcup x) \) pero no siempre \( \bigcup \Set{ \ms{P}(a) \st a \in x } \neq \ms{P}(\bigcup x) \).
      \item \( ( \bigcup x \big) \cap ( \bigcup y ) = \bigcup \Set{ a \cap b \st (a,b) \in x \times y } \).
  \end{enumerate}
\end{exercise}

\begin{exercise}
  Sean $X,Y, \ms{F}$ conjuntos tales que $\ms{F}\neq \emptyset$ y $f\colon X\to Y$ una función. Demuestra que las siguientes clases son conjuntos
  \begin{enumerate}[i)]
      \item $\Class{\bigcup \ms{G}\st \ms{G}\in \ms{F}}$
      \item $\Class{x \st \exists v \exists w \exists y \exists z (v\in \ms{F} \wedge w\in v \wedge y \in w \wedge z \in y \wedge x \in z)}$
      \item $\Class{x \st \forall \ms{G} \in \ms{F} \exists A \in \ms{G} (x\in A)}$
      \item $\Class{\ms{P}(A)\st A\in \ms{F}}$
      \item $\Class{A\times \ms{P}(A)\st A\in \ms{F}}$
      \item $\Class{B\setminus(f[A])\st A\subseteq X \wedge B\in \ms{F}}$
  \end{enumerate}
\end{exercise}

\begin{exercise}[tezfc=1]
  Sean $x$ un conjunto y $f$ una función con dominio $x$. Prueba lo siguiente:
  \begin{enumerate}[i)]
      \item Si $A \in \ms{P}(\ms{P}(X))$ es no vacío, entonces $f[\bigcap A] \subseteq \bigcap \Set{f[a] \st a \in A} $.
      \item $f$ es inyectiva si y sólo si para cada $A \in \ms{P}(\ms{P}(X))$ no vacío se tiene que $\bigcap \Set{f[a] \st a \in A} \subseteq f[\bigcap A]$.
  \end{enumerate}
\end{exercise}

%Ejercicios para el segundo parcial

\begin{exercise}[parzfc=2]
  Dada una relación $R$, demuestra las siguientes equivalencias:
  \begin{enumerate}[i)]
      \item $R$ es reflexiva si y solo si $\Delta_{dom(R)}\subseteq R$.
      \item $R$ es reflexiva en un conjunto $A$ si y solo si $\Delta_A\subseteq R$.
      \item $R$ es simétrica si y solo si $R^{-1}\subseteq R$.
      \item $R$ es transitiva si y solo si $R\circ R\subseteq R$.
      \item $R$ es irreflexiva si y solo si $R\cap \Delta_V = \emptyset$.
      \item $R$ es antisimétrica si y solo si $R\cap R^{-1}\subseteq \Delta_V$.
      \item $R$ es asimétrica si y solo si $R\cap R^{-1}=\emptyset$.
  \end{enumerate}
\end{exercise}

\begin{exercise}[parzfc=2]
  Para cada inciso, da un ejemplo de relacion $R$ (sobre algún conjunto $A$) de modo que:
  \begin{enumerate}[i)]
    \item $R$ sea simétrica y antisimétrica a la vez. ¿Tal relación es única?
    \item $R$ sea reflexiva y antireflexiva a la vez. ¿Tal relación es única?
  \end{enumerate}
\end{exercise}

\begin{exercise}[parzfc=2]
  Si $R$ es un orden parcial sobre $A$, definimos $R^\prime=R\cup \Delta_A$ como el orden parcial reflexivo asociado; por otro lado si $R$ es reflexivo, definimos $R^*=R\setminus \Delta_A$ como su orden estricto asociado.

  Demuestra los siguientes puntos:
  \begin{enumerate}[i)]
      \item Si $A\subseteq B$, entonces $(B\setminus A)\cup A=B$.
      \item $A\cap B=\emptyset$, entonces $(B\cup A)\setminus A=B$.
      \item $R^\prime$ es efectivamente un orden parcial reflexivo sobre $A$.
      \item $R^*$ es efectivamente un orden estricto sobre $A$.
      \item $R^{\prime*}=R$ cuando $R$ es estricto.
      \item $R^{*\prime}=R$ cuando $R$ es reflexivo. Esto junto al inciso anterior prueba que los órdenes estrictos y reflexivos están asociados mediante unaa biyección.
  \end{enumerate}
\end{exercise}


\begin{exercise}[parzfc=2]
  Dadas $R,S$ relaciones transitivas y antisimétricas, definimos $R\sim S$ como ${\exists A (R\bigtriangleup S = \Delta_A)}$. Además, definamos al conjunto $\mathcal{X}_A=\set{R\subseteq A^2\st R \text{ es transitiva y antisimétrica}}$.
  
  Demuestra los siguientes incisos:
  \begin{enumerate}[i)]
      \item $\sim$ es reflexiva, transitiva y simétrica.
      \item $\sim_{|_{\mathcal{X}_A}}$ es una relación de equivalencia sobre $\mathcal{X}_A$. A partir de aquí, nos referiremos a esta relación como $\sim$.
      \item Dada $R\in \mathcal{X}_A$, $([R]_\sim, \subseteq)$ es un retículo completo. Prueba que el infimo y supremo son la intersección y unión respectivamente siempre que el conjunto es no vacío.
      \item Prueba que el mínimo es un orden estricto y que el máximo es un orden reflexivo.
      \item Prueba que el el mínimo y máximo están asociados.
      \item Si $R\sim S$, $aRb$ y $bSc$ entonces $a(R\cap S) c$.
  \end{enumerate}
  
\end{exercise}

\begin{exercise}
  Un morfismo de órdenes $f\colon (A,\leq_A)\to (B,\leq_B)$ es una función creciente, es decir, si $a\leq_A b$ implica $f(a)\leq_B f(b)$. Considerando que los órdenes parciales junto con las funciones crecientes forman una categoría, tenemos una definición de isomorfismo.

  Con esto en cuenta, demuestra o refuta con un contraejemplo la siguiente afirmación: $(A,\leq_A)$ es isomorfo a $(B,\leq_B)$ si y solo si existe un morfismo de orden biyectivo.
\end{exercise}

\begin{exercise}[parzfc=2, tezfc=2]
  Sean $(P,<)$ y $(Q,\sqsubset)$ conjuntos totalmente ordenados. Sea $X:=P \times Q$ y defínase la relación $R$ en $X$ como sigue:
  \[ (p,q) \mathrel{R} (x,y) \quad \text{ si y sólo si } \quad \left( \left( p < x \right) \lor \left( p=x \land q \sqsubset y \right) \right) \]
  Demuestre que $(X,R)$ es un conjunto totalmente ordenado.
\end{exercise}

\begin{exercise}[parzfc=2]
  Sean $(P,<)$ y $(Q,\sqsubset)$ conjuntos totalmente ordenados. Sea $X:=(P \times \Set{0}) \cup (Q \times \Set{1})$ y defínase la relación $R$ en $X$ como sigue:
  \[ (x,s) \mathrel{R} (y,t) \quad \text{ si y sólo si } \left( \left( s=0 \land t =1 \right) \right) \]
  Demuestre que $(X,R)$ es un conjunto totalmente ordenado.
\end{exercise}

\begin{exercise}[parzfc=2]
  Sean $P$ un conjunto y $\leq$ un orden parcial reflexivo en $P$. Encuentra una función $f:P \to \mathscr{P}(P)$ inyectiva de modo que para cualesquiera $p,q\in P$ se tenga $p \leq q$ si y sólo si $f(p) \subseteq f(q)$.
\end{exercise}

\begin{exercise}[parzfc=2, tezfc=2]
  Sean $(P,<)$ y $(Q,\sqsubset)$ conjuntos parcialmente ordenados y $f:P \to Q$ tal que para cualesquiera $x,y \in P$: si $x<y$, entonces $f(x) \sqsubset f(y)$ (estas funciones se llaman \textit{``morfismos de orden''}). Demuestra; o refuta mediante un contraejemplo, las siguientes afirmaciones.
  \begin{enumerate}[i)]
    \item Si $p \in P$ es el mínimo de $(P,<)$, entonces $f(p)$ es el mínimo de $(Q,\sqsubset)$.
    \item Si $p \in P$ $<$-minimal de $P$, entonces $f(p)$ es $\sqsubset$-minimal de $Q$.
    \item Si $f$ es biyectiva, entonces $f^{-1}$ es un morfismo de orden.
    \item Si $(P,<)$ es un conjunto totalmente ordenado, entonces $f$ es inyectiva.
  \end{enumerate}
\end{exercise}

\begin{exercise}[parzfc=2]
  Sean $(A,\leq)$ una reticula y $X$ un conjunto. Se define en $A^X$ la relación:
  \[\preccurlyeq :=\set{(f,g)\in (A^X)^2 \st \forall x(x\in X \rightarrow f(x)\leq g(x))}\]
  Demuestra que $(A^X,\preccurlyeq)$ es una retícula.
\end{exercise}

\begin{exercise}[parzfc=2, tezfc=2]
  Sea $P$ un conjunto. Se dice que un orden parcial (antirreflexivo) $R$ en $P$ es \textit{fuertemente inductivo} si y sólo si se satisface:
  \[ \forall A \subseteq P \left( \forall a \in P \left( R^{-1}[\Set{a}] \subseteq A \rightarrow a \in A \right) \rightarrow P = A \right) \]
  Demuestra que para todo orden parcial (antirreflexivo) $R$ en $P$ son equivalentes:
  \begin{enumerate}[i)]
    \item $R$ es total y fuertemente inductivo.
    \item $R$ es buen orden.
  \end{enumerate}
\end{exercise}