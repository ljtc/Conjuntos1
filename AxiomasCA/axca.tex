\documentclass{article}
\usepackage[spanish,mexico,es-noindentfirst]{babel}
\usepackage{mathtools}
\usepackage{mathrsfs}

\newcommand{\ca}{\mathscr{S}}
\newcommand{\topos}{\mathscr{E}}
\newcommand{\con}{\textbf{Con}}

\title{Axiomas de la categoría de conjuntos abstractos}
\author{}
\date{}

\begin{document}

\maketitle

Primero veremos una lista de axiomas como nos han ido aparenciendo en clase.
Luego, refinaremos un poco la primera lista para evitar tantas redundancias.

\section{Lista burda de axiomas}
\begin{description}
  \item[Axioma 0] \(\ca\) es una categoría.
  \item[Axioma 1] \(\ca\) tiene objeto terminal \(1\).
  \item[Axioma 2] \(1\) es separador.
  \item[Axioma 3] \(\ca\) tiene objeto inicial \(0\).
  \item[Axioma 4] \(0\not\cong 1\).
  \item[Axioma 5] \(\ca\) tiene productos binarios (finitos).
  \item[Axioma 6] \(\ca\) tiene coproductos binarios (finitos).
  \item[Axioma 7] \(\ca\) tiene productos fibrados.
  \item[Axioma 8] \(\ca\) tiene clasificador de subobjetos \(\Omega\).
  \item[Axioma 9] \(\ca\) satisface el axioma de elección.
  \item[Axioma 10] \(\ca\) tiene exponenciales.
  \item[Axioma 11] \(\ca\) es booleana, es decir, \(\Omega=1+1\).
  \item[Axioma 12] \(\ca\) es dos valuada, es decir, \(\Omega\) tiene exactamente
  dos elementos.
\end{description}

\section{Lista refinada de axiomas}
\begin{description}
  \item[Axioma 0] \(\ca\) es una categoría.
  \item[Axioma 1] \(\ca\) tiene límites finitos.
  \item[Axioma 2] \(\ca\) tiene clasificador de subobjetos \(\Omega\).
  \item[Axioma 3] \(\ca\) tiene exponenciales.
  \item[Axioma 4] \(1\) es separador.
  \item[Axioma 5] \(\ca\) satisface el axioma de elección.
  \item[Axioma 6] \(\ca\) es booleana.
  \item[Axioma 7] \(\ca\) es dos valuada.
\end{description}

\section{Comentarios}
En esta última lista, si \(\topos\) satisface los axiomas 0--3, entonces se
llama \emph{topos elemental}. Los axiomas 4--7 son propiedades adicionales que
hacen que \(\topos\) sea un \emph{topos de conjuntos abstractos}.

Como veremos más adelante hay topos cuyos objetos se pueden pensar como
conjuntos variables. Sin embargo, la categoría \(\con\) no se comporta de esta
manera. Así, el axioma 4 es necesario para detener la variación. Los axiomas 6 y
7 hacen que la lógica interna de \(\topos\) sea clásica y con sólo dos valores
de verdad. Finalmente, el axioma 5 es necesario para realizar muchas
construcciones en \(\con\) y es independiente del resto. Por lo tanto, se debe
añadir a la lista de axiomas.

\end{document}