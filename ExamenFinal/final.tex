\documentclass[11pt]{article}
\usepackage[spanish,mexico,shorthands=off]{babel}
\usepackage[letterpaper, DIV=classic]{typearea}
\usepackage[T1]{fontenc}
\usepackage{gfsartemisia}
\let\iint\relax
\let\iiint\relax
\let\iiiint\relax
\let\idotsint\relax
\let\openbox\relax
\usepackage{mathtools}
\usepackage{eulervm}
\usepackage{amssymb}
\usepackage{amsthm}
\usepackage{mathrsfs}
\usepackage{xsim}
\usepackage{tasks}
\usepackage{tikz-cd}
\usepackage{titlesec} %Para cambiar estilo de secciones,subsecciones,etc.
\usepackage{titling} %Para cambiar estilo de título
\usepackage{dashrule} %Para hacer líneas punteadas
\usepackage{comment}

\usepackage{multicol}
\usepackage[shortlabels]{enumitem}
\settasks{label=\textsc{\roman*})}

\SetExerciseParameters{exercise}{
  exercise-name = EJ,
  exercise-heading = \paragraph
}

\DeclareExerciseTagging{teca}
\DeclareExerciseTagging{par}
\DeclareExerciseTagging{tezfc}
\DeclareExerciseTagging{parzfc}
\xsimsetup{collect}

\DeclareExerciseCollection[teca=2]{ca}
\DeclareExerciseCollection[tezfc=2]{zfc}

% Para escribir el "tal que" de los conjuntos
\providecommand\st{\;|\;}

\newcommand\SetSymbol[1][]{%
    \nonscript\:#1\vert
    \allowbreak
    \nonscript\:
    \mathopen{}}
    \DeclarePairedDelimiterX\Set[1]\{\}{%
    \renewcommand\st{\SetSymbol[\delimsize]}
    #1
}
    \DeclarePairedDelimiterX\Class[1]\langle\rangle{%
    \renewcommand\st{\SetSymbol[\delimsize]}
    #1
}

%Comandos que utilizamos
\DeclarePairedDelimiter{\name}{\ulcorner}{\urcorner}
\newcommand{\ev}{\mathrm{ev}}
\newcommand{\id}{\mathrm{id}}
\newcommand{\topos}[1]{\mathscr{#1}}
\newcommand{\cat}[1]{\mathbf{#1}}
\newcommand{\op}{{}^{\mathrm{op}}}
\newcommand{\leqop}{\leq^{\mathrm{op}}}
\newcommand{\set}[1]{\{#1\}}
\newcommand{\ms}[1]{\mathscr{#1}}
\renewcommand{\emptyset}{\varnothing}
\newcommand{\zfc}{\text{\textsf{\small ZFC}}}

\newcommand{\pts}[1]{%
  \ifthenelse{\equal{#1}{1}}{\hfill \textbf{(#1 pt)}}{\hfill\textbf{(#1 pts)}}
}

\newcommand{\noloc}{%
  \nobreak
  \mspace{6mu plus 1mu}
  {:}
  \nonscript\mkern-\thinmuskip
  \mathpunct{}
  \mspace{2mu}
}

\DeclareMathOperator{\ima}{ima}
\DeclareMathOperator{\dom}{dom}

%Estilo de secciones
\titleformat*{\section}{\raggedleft\bfseries\Large}
\titleformat*{\subsection}{\raggedleft\bfseries\large}

%Estilo de título
\pretitle{\begin{center}\fontfamily{lmdh}\huge\bfseries}
\posttitle{\par\end{center}\vskip 0.5em}

\begin{document}    
    \section*{Conjuntos}
    \begin{enumerate}[\bf\text{Ej.} 1.]

    \subsection*{Axiomas Básicos}
    \item Demuestra que el enunciado $\varphi \leftrightharpoons \forall x\exists y (x\in y \wedge \forall z \forall w ((z\in w \wedge w\in y) \rightarrow z\in y))$ implica el axioma de unión.
    \textit{Indica claramente cuáles axiomas de $\zfc$ utilizas en la demostración.}

    \[ \varphi \leftrightharpoons \forall x\exists y \forall z \forall w ((z\in w \wedge w\in x) \rightarrow z\in y) \]

    \item Sea $\psi(x)$ una fórmula de la teoría de conjuntos. Demuestre que, si $z:=\{ y \st \exists x \: (\psi(x) \to y \in x) \}$ es un conjunto, entonces $\{x \st \psi(x)\}$ es un conjunto.
    \textit{Indica claramente cuáles axiomas de $\zfc$ utilizas en la demostración.}

    \subsection*{Relaciones y Funciones}
    \item Demuestra que toda relación es una unión de funciones.
    \item Sean $f:\omega \to 2$ y $P:=\{ f \upharpoonright n \in \mathscr{P}(\omega \times 2) \st n \in \omega \}$. Demuestra que $(P,\subseteq)$ es un conjunto bien ordenado.
    
    \subsection*{Dominancia y CSB}
    \item Prueba que $2^\omega$ y $\omega ^ \omega$ son equipotentes.
    
    \item Sean $(B,<)$ un conjunto bien ordenado y $x,A$ conjuntos con $x \subseteq A$. Prueba que, si existe $f:B \to A$ sobreyectiva, entonces $x \preccurlyeq B$.
    
    \subsection*{Copos}
    \item Sea $(A,<)$ una retícula (latiz). Demuestra que si $(A,<)$ no es distributiva, entonces existe un subconjunto $B=\{a,b,c,d,e\} \subseteq A$ de modo que $< \upharpoonright B$ es alguno de los siguientes:
    \begin{enumerate}
        \item Diamante: $\{(a,b),(b,e),(a,c),(c,e),(a,d),(d,e),(a,e)\}$
        \item Pentagono: $\{(a,e),(a,d),(d,e),(a,c),(c,e),(a,b),(b,c),(b,e)\}$
    \end{enumerate}

    \item Pruebe que si un orden parcial $(P,<)$ es fuertemente inductivo, entonces cada $A \subseteq P$ no vacío posee un $<$-minimal.
    
    \section*{Naturales e Inducción}
    \item Sean $f:X\to \omega$ y $Y\subseteq X$ cualesquiera. Demuestra que si para cada $x\in X$ se satisface la proposición: $\forall y\in X (f(y)<f(x) \to y\in Y) \to x\in Y$, entonces $Y=X$.
    
    \item Un conjunto $X$ es \text{Tarski-finito} si y sólo si para cada $A \subseteq X$ no vacío, existe $y \in A$ de modo que para cada $a \in A$, no ocurre $y \subsetneq a$. Demuestra que todo natural $n \in \omega$ es Tarski-finito.
   

    \item Sea $X$ un conjunto. Una $\in,X$-cadena es una función $f$ con dominio algún natural $n \in \omega$ que cumple: $f(0) \in X$; y, para cualesquiera $m,k \in n$ con $m<k$, se cumple $f(k) \in f(m)$. 
    
    Es un hecho que $C=\bigcup \{ \ima(f) \st f \text{ es} \in,X \text{ cadena}\}$ es un conjunto. Pruebe que $C$ es un conjunto transitivo tal que $X \subseteq C$.

    \subsection*{Recursión}
    \item Utilizando \textit{únicamente} el Primer Teorema de Recursión (1TR), demuestre que existe una función $F:\omega \to \omega$ de modo que $F(0)=1$; y, para cada $n \in \omega$, $F(s(n))=s(n) \cdot F(n)$.
    
    \hfill \textit{Hint: Considere $X:=\omega \times \omega$ en el 1TR, con el punto inicial $(1,1)$ y defina una dinámica adecuada $g:X \to X$ de modo que al proyectar $g$ a la primera entrada, consiga la función $F$.}
    
    \item Sean $X$ un conjunto y $f:X \to X$. Demuestre que existe una función $g:\omega \times X \to X$ de modo que para cada $(n,x) \in \omega \times X$ se cumplen $g(0,x)=x$ y $g(s(n),x)=g(n,f(x))$.

    \item Sea $f\colon X\to X$ una función y $A\subseteq X$. Consideremos, mediante el teorema de recursión, la  (única) función $g\colon N\to \mathscr{P}(X)$ de modo que $g(0) = A$ y, para cada $n \in \omega$, $g(s(n)) = g(n)\cup f[g(n)]$.
    Definimos a los conjuntos $A_*=\bigcup im(g)$, y $A^*= \bigcap\set{B\subseteq X \st A\subseteq B \wedge f[B]\subseteq B}$ (\textit{llamados las cerraduras inferiores y superiores de $A$ bajo $f$, respectivamente}). Demuestra que:
    $$ A^*=A_* \text{ ,} \quad A\subseteq A^* \quad \text{y} \quad f[A^*]\subseteq A^* $$



    \end{enumerate}
\end{document}