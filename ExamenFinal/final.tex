\documentclass[11pt]{article}
\usepackage[spanish,mexico,shorthands=off]{babel}
\usepackage[letterpaper, DIV=classic]{typearea}
\usepackage[T1]{fontenc}
\usepackage{gfsartemisia}
\let\iint\relax
\let\iiint\relax
\let\iiiint\relax
\let\idotsint\relax
\let\openbox\relax
\usepackage{mathtools}
\usepackage{eulervm}
\usepackage{amssymb}
\usepackage{amsthm}
\usepackage{mathrsfs}
\usepackage{xsim}
\usepackage{tasks}
\usepackage{tikz-cd}
\usepackage{titlesec} %Para cambiar estilo de secciones,subsecciones,etc.
\usepackage{titling} %Para cambiar estilo de título
\usepackage{dashrule} %Para hacer líneas punteadas
\usepackage{comment}

\usepackage{multicol}
\usepackage[shortlabels]{enumitem}
\settasks{label=\textsc{\roman*})}

\SetExerciseParameters{exercise}{
  exercise-name = EJ,
  exercise-heading = \paragraph
}

\DeclareExerciseTagging{teca}
\DeclareExerciseTagging{par}
\DeclareExerciseTagging{tezfc}
\DeclareExerciseTagging{parzfc}
\xsimsetup{collect}

\DeclareExerciseCollection[teca=2]{ca}
\DeclareExerciseCollection[tezfc=2]{zfc}

% Para escribir el "tal que" de los conjuntos
\providecommand\st{\;|\;}

\newcommand\SetSymbol[1][]{%
    \nonscript\:#1\vert
    \allowbreak
    \nonscript\:
    \mathopen{}}
    \DeclarePairedDelimiterX\Set[1]\{\}{%
    \renewcommand\st{\SetSymbol[\delimsize]}
    #1
}
    \DeclarePairedDelimiterX\Class[1]\langle\rangle{%
    \renewcommand\st{\SetSymbol[\delimsize]}
    #1
}

%Comandos que utilizamos
\DeclarePairedDelimiter{\name}{\ulcorner}{\urcorner}
\newcommand{\ev}{\mathrm{ev}}
\newcommand{\id}{\mathrm{id}}
\newcommand{\topos}[1]{\mathscr{#1}}
\newcommand{\cat}[1]{\mathbf{#1}}
\newcommand{\op}{{}^{\mathrm{op}}}
\newcommand{\leqop}{\leq^{\mathrm{op}}}
\newcommand{\set}[1]{\{#1\}}
\newcommand{\ms}[1]{\mathscr{#1}}
\renewcommand{\emptyset}{\varnothing}
\newcommand{\zfc}{\text{\textsf{\small ZFC}}}

\newcommand{\pts}[1]{%
  \ifthenelse{\equal{#1}{1}}{\hfill \textbf{(#1 pt)}}{\hfill\textbf{(#1 pts)}}
}

\newcommand{\noloc}{%
  \nobreak
  \mspace{6mu plus 1mu}
  {:}
  \nonscript\mkern-\thinmuskip
  \mathpunct{}
  \mspace{2mu}
}

\DeclareMathOperator{\ima}{ima}
\DeclareMathOperator{\dom}{dom}

%Estilo de secciones
\titleformat*{\section}{\raggedleft\bfseries\Large}
\titleformat*{\subsection}{\raggedleft\bfseries\large}

%Estilo de título
\pretitle{\begin{center}\fontfamily{lmdh}\huge\bfseries}
\posttitle{\par\end{center}\vskip 0.5em}

\begin{document}    
    \section*{Conjuntos}
    \begin{enumerate}[\bf\text{Ej.} 1.]

    \subsection*{Axiomas Básicos}
    \item Demuestra que el enunciado $\varphi \leftrightharpoons \forall x\exists y (x\in y \wedge \forall z \forall w ((z\in w \wedge w\in y) \rightarrow z\in y))$ implica el axioma de unión.
    \textit{Indica claramente cuáles axiomas de $\zfc$ utilizas en la demostración.}

    \item Sea $\psi(x)$ una fórmula de la teoría de conjuntos. Demuestre que, si $z:=\{ y \st \exists x \: (\psi(x) \to y \in x) \}$ es un conjunto, entonces $\{x \st \psi(x)\}$ es un conjunto.
    \textit{Indica claramente cuáles axiomas de $\zfc$ utilizas en la demostración.}

    \subsection*{Relaciones y Funciones}
    \item Demuestra que toda relación es una unión de funciones.
    \item Sean $f:\omega \to 2$ y $P:=\{ f \upharpoonright n \in \mathscr{P}(\omega \times 2) \st n \in \omega \}$. Demuestra que $(P,\subseteq)$ es un conjunto bien ordenado.
    
    \subsection*{Dominancia y CSB}
    \item Prueba que $2^\omega$ y $\omega ^ \omega$ son equipotentes.
    
    \item Sean $(B,<)$ un conjunto bien ordenado y $x,A$ conjuntos con $x \subseteq A$. Prueba que, si existe $f:B \to A$ sobreyectiva, entonces $x \preccurlyeq B$.
    
    \subsection*{Copos}
    \item Sea $(A,<)$ una retícula (latiz). Demuestra que si $(A,<)$ no es distributiva, entonces existe un subconjunto $B=\{a,b,c,d,e\} \subseteq A$ de modo que $< \upharpoonright B$ es alguno de los siguientes:
    \begin{enumerate}
        \item Diamante: $\{(a,b),(b,e),(a,c),(c,e),(a,d),(d,e),(a,e)\}$
        \item Pentagono: $\{(a,e),(a,d),(d,e),(a,c),(c,e),(a,b),(b,c),(b,e)\}$
    \end{enumerate}

    \item Pruebe que si un orden parcial $(P,<)$ es fuertemente inductivo, entonces cada $A \subseteq P$ no vacío posee un $<$-minimal.
    
    \section*{Naturales e Inducción}
    \item Sean $f:X\to \omega$ y $Y\subseteq X$ cualesquiera. Demuestra que si para cada $x\in X$ se satisface la proposición: $\forall y\in X (f(y)<f(x) \to y\in Y) \to x\in Y$, entonces $Y=X$.
    
    \item Un conjunto $X$ es \text{Tarski-finito} si y sólo si para cada $A \subseteq X$ no vacío, existe $y \in A$ de modo que para cada $a \in A$, no ocurre $y \subsetneq a$. Demuestra que todo natural $n \in \omega$ es Tarski-finito.
   

    \item Sea $X$ un conjunto. Una $\in,X$-cadena es una función $f$ con dominio algún natural $n \in \omega$ que cumple: $f(0) \in X$; y, para cualesquiera $m,k \in n$ con $m<k$, se cumple $f(k) \in f(m)$. 
    
    Es un hecho que $C=\bigcup \{ \ima(f) \st f \text{ es} \in,X \text{ cadena}\}$ es un conjunto. Pruebe que $C$ es un conjunto transitivo tal que $X \subseteq C$.

    \subsection*{Recursión}
    \item Utilizando \textit{únicamente} el Primer Teorema de Recursión (1TR), demuestre que existe una función $F:\omega \to \omega$ de modo que $F(0)=1$; y, para cada $n \in \omega$, $F(s(n))=s(n) \cdot F(n)$.
    
    \hfill \textit{Hint: Considere $X:=\omega \times \omega$ en el 1TR, con el punto inicial $(1,1)$ y defina una dinámica adecuada $g:X \to X$ de modo que al proyectar $g$ a la primera entrada, consiga la función $F$.}
    
    \item Sean $X$ un conjunto y $f:X \to X$. Demuestre que existe una función $g:\omega \times X \to X$ de modo que para cada $(n,x) \in \omega \times X$ se cumplen $g(0,x)=x$ y $g(s(n),x)=g(n,f(x))$.

    \item Sea $f\colon X\to X$ una función y $A\subseteq X$. Consideremos, mediante el teorema de recursión, la  (única) función $g\colon N\to \mathscr{P}(X)$ de modo que $g(0) = A$ y, para cada $n \in \omega$, $g(s(n)) = g(n)\cup f[g(n)]$.
    Definimos a los conjuntos $A_*=\bigcup im(g)$, y $A^*= \bigcap\set{B\subseteq X \st A\subseteq B \wedge f[B]\subseteq B}$ (\textit{llamados las cerraduras inferiores y superiores de $A$ bajo $f$, respectivamente}). Demuestra que:
    $$ A^*=A_* \text{ ,} \quad A\subseteq A^* \quad \text{y} \quad f[A^*]\subseteq A^* $$
    \end{enumerate}


  \section*{Conjuntos Abstractos}
  Los ejercicios de esta sección se deben resolver en la categoría de conjuntos
  abstractos, \(\topos{S}\), a menos que se indique lo contrario.
  \begin{enumerate}[\bf\text{Ej.} 1.]
    \item Sea \(f\colon A\to B\). Definimos la gráfica de \(f\) como
      \begin{equation*}
        \begin{tikzcd}
          A & A\times B\ar{l}[swap]{p_A}\ar{r}{p_B} & B\\
          & A\mathrlap{.}\ar{ul}{\id}\ar{ur}[swap]{f}\ar[dashed]{u}{G_f}
        \end{tikzcd}
      \end{equation*}
      Demuestra que \(G_f\) es mono.

    \item Sean \(f,g\colon A\to B\) tales que \(G_f=G_g\). Demuestra que \(f=g\).
    
    \item Demuestra que \(A+0\cong A\).
    
    \item Demuestra que toda flecha de la forma \(0\to A\) es inyectiva (por lo
    tanto mono). Además muestra que si \(A\) no tiene un elementos globales, entonces
    \(0\to A\) es biyectiva (por lo tanto iso).
    
    \item Una flecha \(f\colon A\to B\) es constante si se factoriza a través de
    \(1\), es decir, existe \(b\colon 1\to B\) tal que el siguiente diagrama
    conmuta:
    \begin{equation*}
      \begin{tikzcd}
        A\ar{rr}{f}\ar{dr}[swap]{!_A} && B\\
        & 1\mathrlap{.}\ar{ur}[swap]{b}
      \end{tikzcd}
    \end{equation*}
    Demuestra que \(f\) es constante si y sólo si para cualesquiera 
    \(a_1,a_2\colon 1\to A\) se satisface \(fa_1 = fa_2\).

    \item Demuestra que las inclusiones a un coproducto son monos. Esto es, si 
    \(i_A\colon A\to A+B\leftarrow B\noloc i_B\) es un diagrama de coproducto,
    entonces \(i_A\) es mono.

    \item Sea \(m\colon T\rightarrowtail B\) un subobjeto y \(f\colon A\to B\).
    Encuentra la flecha característica de la imagen inversa de \(m\).

    \item Sea \(\cat{A}\) una categoría. Considera flechas \(f\colon A\to B\) y
    \(r,s\colon B\to A\) tales que \(rf=\id_A\) y \(fs=\id_B\). Demuestra que
    \(r=s\).

    \item Sea \(\cat{A}\) una categoría. Si \(fs=\id_B\) y \(f\) es mono,
    entonces \(s\) es inversa, por los dos lados, de \(f\).

    \item Sea \(f\colon A\to B\). La fibra de \(b\colon 1\to B\) es el producto fibrado
    \begin{equation*}
      \begin{tikzcd}
        f^-1(b)\ar{r}[swap]{m}\ar[tail]{d} & 1\ar{d}{b}\\
        A\ar{r}[swap]{f} & B\mathrlap{.}
      \end{tikzcd}
    \end{equation*}
    Demuestra que si \(s\) es una sección de \(f\), entonces \(sb\in_A m\), pra
    todo \(b\colon 1\to B\).

    \item Demuestra las siguientes leyes exponenciales:
  \begin{tasks}(2) \task \(A^0=1\) \task \(A^1=A\) \task \(A^{B+C}\cong
    A^B\times A^C\) \task \(A^{B\times C}\cong (A^B)^C\)
  \end{tasks}

  \item Sea \(\cat{A}\) una categoría localmente pequeña con coproductos. Demuestra que
  \begin{equation*}
    \cat{A}(A,C)\times\cat{A}(B,C)\cong\cat{A}(A+B,C).
  \end{equation*}

  \item Demuestra que \(f\colon A\to B\) es epi si y sólo si 
  \(\Omega^f\colon \Omega^B\to \Omega^A\) es mono.
    
  \end{enumerate}
\end{document}