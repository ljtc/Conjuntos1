\documentclass{article}
\usepackage{mathtools}
\usepackage{kantlipsum}
\usepackage{xsim}

\SetExerciseParameters{exercise}{
  exercise-name = Ejercicio,
  exercise-heading = \paragraph
}

\newcommand{\topos}[1]{\mathcal{#1}}

\title{Tarea n}
\author{Teoría de Conjuntos 1}
\date{\today}

\begin{document}
\maketitle

\section{ZFC}


\section{Conjuntos Abstractos}
\begin{exercise}
    Muestra que toda flecha de la forma \(1\to A\) (con dominio el terminal) es mono.
\end{exercise}

\begin{exercise}
    Demuestra que si \(T\) tiene un elemento global entonces es separador.
\end{exercise}

\begin{exercise}
    Demuestra, en \(\topos{S}\), que mono es equivalente a inyectiva.
\end{exercise}

\end{document}