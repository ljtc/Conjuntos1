\documentclass[12pt]{article}
\usepackage[spanish,mexico,shorthands=off]{babel}
\usepackage[letterpaper, DIV=classic]{typearea}
\usepackage[T1]{fontenc}
\usepackage{gfsartemisia}
\let\iint\relax
\let\iiint\relax
\let\iiiint\relax
\let\idotsint\relax
\let\openbox\relax
\usepackage{mathtools}
\usepackage{eulervm}
\usepackage{amssymb}
\usepackage{amsthm}
\usepackage{mathrsfs}
\usepackage{xsim}
\usepackage{tasks}
\usepackage{tikz-cd}
\usepackage{titlesec} %Para cambiar estilo de secciones,subsecciones,etc.
\usepackage{titling} %Para cambiar estilo de título
\usepackage{dashrule} %Para hacer líneas punteadas
\usepackage{comment}

\usepackage{multicol}
\usepackage[shortlabels]{enumitem}
\settasks{label=\textsc{\roman*})}

\SetExerciseParameters{exercise}{
  exercise-name = EJ,
  exercise-heading = \paragraph
}

\DeclareExerciseTagging{teca}
\DeclareExerciseTagging{par}
\DeclareExerciseTagging{tezfc}
\DeclareExerciseTagging{parzfc}
\xsimsetup{collect}

\DeclareExerciseCollection[teca=2]{ca}
\DeclareExerciseCollection[tezfc=2]{zfc}

% Para escribir el "tal que" de los conjuntos
\providecommand\st{\;|\;}

\newcommand\SetSymbol[1][]{%
    \nonscript\:#1\vert
    \allowbreak
    \nonscript\:
    \mathopen{}}
    \DeclarePairedDelimiterX\Set[1]\{\}{%
    \renewcommand\st{\SetSymbol[\delimsize]}
    #1
}
    \DeclarePairedDelimiterX\Class[1]\langle\rangle{%
    \renewcommand\st{\SetSymbol[\delimsize]}
    #1
}

%Comandos que utilizamos
\DeclarePairedDelimiter{\name}{\ulcorner}{\urcorner}
\newcommand{\ev}{\mathrm{ev}}
\newcommand{\id}{\mathrm{id}}
\newcommand{\topos}[1]{\mathscr{#1}}
\newcommand{\cat}[1]{\mathbf{#1}}
\newcommand{\op}{{}^{\mathrm{op}}}
\newcommand{\leqop}{\leq^{\mathrm{op}}}
\newcommand{\set}[1]{\{#1\}}
\newcommand{\ms}[1]{\mathscr{#1}}
\renewcommand{\emptyset}{\varnothing}
\newcommand{\zfc}{\text{\textsf{\small ZFC}}}

\newcommand{\pts}[1]{%
  \ifthenelse{\equal{#1}{1}}{\hfill \textbf{(#1 pt)}}{\hfill\textbf{(#1 pts)}}
}

\newcommand{\noloc}{%
  \nobreak
  \mspace{6mu plus 1mu}
  {:}
  \nonscript\mkern-\thinmuskip
  \mathpunct{}
  \mspace{2mu}
}

\DeclareMathOperator{\ima}{ima}
\DeclareMathOperator{\dom}{dom}

%Estilo de secciones
\titleformat*{\section}{\raggedleft\bfseries\Large}
\titleformat*{\subsection}{\raggedleft\bfseries\large}

%Estilo de título
\pretitle{\begin{center}\fontfamily{lmdh}\huge\bfseries}
\posttitle{\par\end{center}\vskip 0.5em}

\begin{document}   

  \begin{center}
   \Huge \textbf{Idea de Final} \\
  \end{center}

  \begin{flushright}
   \footnotesize Profesor: Luis Jesús Trucio Cuevas. \hfill Ayudantes: Jesús Angel Cabrera Labastida,\\
   \hfill Hugo Víctor García Martínez.
  \end{flushright}

  \section*{Conjuntos Abstractos}
    Los ejercicios de esta sección se deben resolver en la categoría de conjuntos abstractos, \(\topos{S}\), a menos que se indique lo contrario.

    \begin{enumerate}[\bf\text{Ej.} 1.]
    \item Pruebe que $\ms{S}$ es una categoría balanceada; esto es, para cualquier flecha $f:A \to B$ se tiene que $f$ es isomorfismo si y solo si $f$ es monomorfismo y epimorfismo.
    
    \item Sean $\ms{C}$ una categoría localamente pequeña y $A,B$ objetos de $\ms{C}$. Utilizando el Lema de Grothendieck - Yoneda, demuestre que si $\ms{C}(-,A) \cong \ms{C}(-,B)$, entonces $A \cong B$.
    
    \item Sea $(A \xrightarrow{f} B) \in \ms{S}$. Demuestre que $f$ es epimorfismo si y sólo si $\Omega^f$ es monomorfismo.
    
    
    \end{enumerate}

    \section*{ZFC}
    Resuelva los siguientes ejercicios utilizando los axiomas de $\zfc$ vistos en clase (Vacío, Extensionalidad, Par, Unión, Esq. de Separación, Potencia, Infinito)

    \begin{enumerate}[\bf\text{Ej.} 1.]
    \setcounter{enumi}{3}

    \item Demuestra que el enunciado $\varphi \leftrightharpoons \forall x\exists y (x\in y \wedge \forall z \forall w ((z\in w \wedge w\in y) \rightarrow z\in y))$ implica el axioma de unión.
    
    \item Sea $X$ un conjunto de números naturales. Determine cuáles de las siguientes implicaciones son verdaderas. Justifique con demostración o contraejemplo.
    \begin{enumerate}
        \item Si $X$ es transitivo, entonces $X$ es natural.
        \item Si $X$ es no vacío, entonces $\bigcap X = \min_\in (X)$.
    \end{enumerate}

    \item Demuestre que hay una biyección entre $3^\omega$ y $\omega^\omega$.

    \end{enumerate}
  
\end{document}